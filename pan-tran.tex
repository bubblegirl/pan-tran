\documentclass[12pt]{article}
 \usepackage[hcentering,bindingoffset=20mm]{geometry}
 \usepackage{placeins}
 \usepackage[numbib]{tocbibind}
 \usepackage{rotating}
\usepackage[square,sort,comma,numbers]{natbib}
 \usepackage{graphicx}
 \usepackage{tabularx}
 \linespread{1.3}
 \usepackage{gensymb}
\usepackage{longtable}
 \usepackage{lscape}
 \usepackage{url}
 \addtolength{\textwidth}{2cm}
 \addtolength{\hoffset}{-1cm}
 
 
 \addtolength{\textheight}{2cm}
 \addtolength{\voffset}{-1cm}
 \setlength{\parindent}{0pt}
 
\title{Pan-transcriptomics of \emph{Gambierdiscus}, with a focus on \emph{G. polynesiensis}. $^{1}$}
\author{Key words: \emph{Gambierdiscus}, ciguatoxin, pan-transcriptome}
\date{}

\begin{document}
\maketitle
\paragraph{}Anna Liza Kretzschmar$^{2}$\\
Climate Change Cluster (C3), University of Technology Sydney, Ultimo, 2007 NSW, Australia, anna.kretzschmar@uts.edu.au
\paragraph{}Tim Kahlke\\
Climate Change Cluster (C3), University of Technology Sydney, Ultimo, 2007 NSW, Australia
\paragraph{}Kirsty Smith \\
Cawthron Institute, The Wood, Nelson 7010, New Zealanda
\paragraph{}Lesley Rhodes \\
Cawthron Institute, The Wood, Nelson 7010, New Zealand
\paragraph{}Aaron E. Darling \\
The ithree institute, University of Technology Sydney, Ultimo, 2007 NSW, Australia
\paragraph{}Shauna Murray\\ 
Climate Change Cluster (C3), University of Technology Sydney, Ultimo, 2007 NSW, Australia
\newpage
\section*{Abstract}



Species of the genus \textit{Gambierdiscus} produce Ciguatoxins (CTXs), the causative agent of ciguatera fish poisoning, a potentially debilitating seafood borne illness. 
Species of \textit{Gambierdiscus} possess very large genomes, ~35 Gbp, and, as with other dinoflagellates, possess unique genomic characteristics, e.g. highly repetitive and complex genome architecture, meaning that genomic sequencing remains infeasible. 
While \textit{G. polyneisensis} is the only species verified as CTX producers (through LC-MS/MS), others produce different types of toxins such as Maitotoxins, Gambierol, and other uncharacterized compounds. 
Understanding the evolution of Gambierdiscus and their toxins requires knowledge about their genetic and metabolic potential. 
Transcriptomic approaches circumvent the challenges of whole genome sequencing and enable the investigation of the core-transcriptome of \textit{Gambierdiscus} species. 
In this study, we generated de novo RNA-seq libraries for \textit{Gambierdiscus polynesiensis}, \textit{Gambierdiscus carpenteri}, \textit{Gambierdiscus} cf. \textit{silvae} and \textit{Gambierdiscus lapillus}, compared these to a previously sequenced \textit{Gambierdiscus australes}, to discover a set of core genes shared by all species. 
We present a Gambierdiscus core transcriptome, which might be used to investigate candidate genes related to toxin production.\\
Further to investigate CTX production more specifically, we compared two CTX -producing strains of \textit{Gambierdiscus polynesiensis} to one non-CTX producing strain, verified by LC-MS/MS, to look for clues about pathways involved in ciguatoxin production.

\newpage
\section*{Introduction}
-protist genome sizes, obstacles with sequencing, lack of reference genomes available \\
- transcriptomes as alternative, dino genetic elements, transcriptomes a good stand in for ref genome\\
- Gambierdiscus intro, clinical relevance of CFP, interest in monitoring\\
- G. polynesiensis and implications in CFP specifically, CTX and associated search for PKS genes\\
- focus of study: core-transcriptome for \textit{Gambierdiscus} for RNA-seq reference purposes and polynesiensis comparison to look for expression differences between toxic and non-toxic strains
\newpage
\section*{Methods}

\subsection*{Culture conditions}
%TODO Kirsty to insert method
\subsection*{RNA isolation}
%TODO Kirsty to insert method
\subsection*{Library prep and sequencing}
%TODO Kirsty to insert method
\subsection*{Transcriptome assembly \& annotation}

\subsection*{Publicly available transcriptomes}

\subsection*{Pan-transcriptome generation}

\subsection*{\textit{G. polynesiensis} comparison}
\newpage
\section*{Results}
\FloatBarrier
\subsection*{Transcriptome overview}
- seq and annotation stats for \textit{Gambierdiscus polynesiensis} CAWD254 and Table ~\ref{tbl:SeqTable}
\subsection*{\emph{Gambierdiscus} inter-species core \& pan transcriptome}

\subsection*{\emph{Gambierdiscus polynesiensis} intra-species core \& pan transcriptome}

\subsubsection*{Polyketide synthase genes}
\begin{table}
\caption{\emph{Gambierdiscus} species transcriptomes used in this study along with their toxicity, toxin profile, accession numbers and source.}
\label{tbl:SeqTable}
\begin{tabular}{ | p{3cm} | p{2cm} | p{2.5cm} | p{2.5cm} | p{2cm} | p{2cm}|}
\hline
\textbf{Species} & \textbf{Strain}& \textbf{LC-MS/MS} & \textbf{N2A
} & \textbf{Accession ID} & \textbf{References} \\
\hline
\textit{Gambierdiscus australes}&CAWD149& CTX -ve; MTX +ve&CTX +ve; MTX N/A&MMETSP0766&\cite{keeling2014marine,rhodes2010toxic,rhodes2014production}\\
\hline
\textit{Gambierdiscus carpenteri}&UTSMER9A&CTX -ve; MTX -ve&CTX -ve; MTX +ve&SRR6821720
&Kretzschmar '18 in prep, \cite{larsson2018toxicology}\\
\hline
\textit{Gambierdiscus lapillus}&HG4&CTX -ve; MTX +ve&CTX +ve; MTX +ve&SRR6821722
&Kretzschmar '18 in prep, \cite{larsson2018toxicology,kretzschmar2017characterization}\\
\hline
\textit{Gambierdiscus polynesiensis}&CG15&CTX +ve; MTX +ve&CTX N/A; MTX N/A&SRR6821723
&Kretzschmar '18 in prep\\
\hline
\textit{Gambierdiscus polynesiensis}&CAWD212&CTX +ve; MTX +ve&CTX N/A; MTX N/A&Kohli pers. comm.&\cite{rhodes2014production}\\
\hline
\textit{Gambierdiscus polynesiensis}&CAWD254&CTX -ve; MTX -ve&CTX N/A; MTX N/A&&This study\\
\hline
\textit{Gambierdiscus} cf. \textit{silvae}&HG5&CTX -ve; MTX +ve&CTX +ve; MTX +ve&SRR6821721
&Kretzschmar '18 in prep, \cite{larsson2018toxicology,kretzschmar2017characterization}\\
\hline
\end{tabular}
\end{table}
\FloatBarrier
\newpage
\section*{Discussion}
-overall summary of study
\subsection*{core \textit{Gambierdiscus} transcriptome}
-discuss common gene pathways found\\
-discuss usefulness for future studies\\
-discuss potential short commings from different seq runs and methods and number of species used with only one rep per species

\subsection*{core \textit{G. polynesiensis} transcriptome}
- discuss common gene pathways found \\
-discuss usefulness for future studies\\
-discuss potential short commings
\subsubsection*{Expression of genes involved in polyketide production}
- discuss if different gene sets were expressed between toxic and non- toxic strains\\
- discuss if diff genes between all 3 strains\\
- discuss short comings with CAWD254 not having bioassays run, sequencing differneces etc

%\newpage
\section*{Conclusion}


\newpage
\bibliographystyle{acm}
\bibliography{pantran.bib}


\end{document}