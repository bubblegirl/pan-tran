\documentclass[12pt]{article}
 \usepackage[hcentering,bindingoffset=20mm]{geometry}
 \usepackage{placeins}
 \usepackage[numbib]{tocbibind}
 \usepackage{rotating}
\usepackage[square,sort,comma,numbers]{natbib}
 \usepackage{graphicx}
 \usepackage{tabularx}
 \linespread{1.3}
 \usepackage{gensymb}
\usepackage{longtable}
 \usepackage{lscape}
 \usepackage{url}
 \addtolength{\textwidth}{2cm}
 \addtolength{\hoffset}{-1cm}
 
 
 \addtolength{\textheight}{2cm}
 \addtolength{\voffset}{-1cm}
 \setlength{\parindent}{0pt}
 
\title{Pan-transcriptomics of \emph{Gambierdiscus}, with a focus on \emph{G. polynesiensis}. $^{1}$}
\author{Key words: \emph{Gambierdiscus}, ciguatoxin, pan-transcriptome}
\date{}

\begin{document}
\maketitle
\paragraph{}Anna Liza Kretzschmar$^{2}$\\
Climate Change Cluster (C3), University of Technology Sydney, Ultimo, 2007 NSW, Australia, anna.kretzschmar@uts.edu.au
\paragraph{}Tim Kahlke\\
Climate Change Cluster (C3), University of Technology Sydney, Ultimo, 2007 NSW, Australia
\paragraph{}Kirsty Smith \\
Cawthron Institute, The Wood, Nelson 7010, New Zealanda
\paragraph{}Lesley Rhodes \\
Cawthron Institute, The Wood, Nelson 7010, New Zealand
\paragraph{}Aaron E. Darling \\
The ithree institute, University of Technology Sydney, Ultimo, 2007 NSW, Australia
\paragraph{}Shauna Murray\\ 
Climate Change Cluster (C3), University of Technology Sydney, Ultimo, 2007 NSW, Australia
\newpage
\section*{Abstract}

Species of the genus Gambierdiscus produce Ciguatoxins (CTXs), the causative agent of ciguatera fish poisoning, a potentially debilitating seafood borne illness. Species of Gambierdiscus possess very large genomes, ~35 Gbp, and, as with other dinoflagellates, possess unique genomic characteristics, e.g. highly repetitive and complex genome architecture, meaning that genomic sequencing remains infeasible. While G. polyneisensis is the only species verified as CTX producers (through LC-MS/MS), others produce different types of toxins such as Maitotoxins, Gambierol, and other uncharacterized compounds. Understanding the evolution of Gambierdiscus and their toxins requires knowledge about their genetic and metabolic potential. Transcriptomic approaches circumvent the challenges of whole genome sequencing and enable the investigation of the core-transcriptome of Gambierdiscus species. In this study, we generated de novo RNA-seq libraries for  Gambierdiscus polynesiensis, Gambierdiscus carpenteri, Gambierdiscus cf. silvae and Gambierdiscus lapillus, compared these to a previously sequenced Gambierdiscus australes, to discover a set of core genes shared by all species. We present a Gambierdiscus core transcriptome, which might be used to investigate candidate genes related to toxin production.

Further to investigate CTX production more specifically, we compared two CTX -producing strains of Gambierdiscus polynesiensis to one non-CTX producing strain, verified by LC-MS/MS, to look for clues about pathways involved in ciguatoxin production.

\newpage
\section*{Introduction}

\newpage
\section*{Methods}

\newpage
\section*{Results}

\newpage
\section*{Discussion}

\newpage
\section*{Conclusion}

\newpage
\section*{References}

\newpage
\bibliographystyle{acm}
\bibliography{pantran.bib}


\end{document}